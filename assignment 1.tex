\documentclass[]{article}

%%%%%%%%%%%%%%%%%%%
% Packages/Macros %
%%%%%%%%%%%%%%%%%%%
\usepackage{amssymb,latexsym,amsmath}     % Standard packages


%%%%%%%%%%%
% Margins %
%%%%%%%%%%%
\addtolength{\textwidth}{1.0in}
\addtolength{\textheight}{1.00in}
\addtolength{\evensidemargin}{-0.75in}
\addtolength{\oddsidemargin}{-0.75in}
\addtolength{\topmargin}{-.50in}


%%%%%%%%%%%%%%%%%%%%%%%%%%%%%%
% Theorem/Proof Environments %
%%%%%%%%%%%%%%%%%%%%%%%%%%%%%%
\newtheorem{theorem}{Theorem}
\newenvironment{proof}{\noindent{\bf Proof:}}{$\hfill \Box$ \vspace{10pt}}  


%%%%%%%%%%%%
% Document %
%%%%%%%%%%%%
\begin{document}

\title{THE EFFECTS OF STRESS ON BUSINESS EMPLOYEES
	AND PROGRAMS OFFERED BY EMPLOYERS
	TO MANAGE EMPLOYEE STRESS }
\author{Kato Fahad Twahir}
\maketitle

\section{Introduction}
	Today, many organizations and employees are experiencing the effects of stress on
work performance. The effects of stress can be either positive or negative. What is
perceived as positive stress by one person may be perceived as negative stress by
another, since everyone perceives situations differently. According to Barden (2001),
negative stress is becoming a major illness in the work environment, and it can
debilitate employees and be costly to employers. Managers need to identify those
suffering from negative stress and implement programs as a defense against stress.
These programs may reduce the impact stress has on employees' work performance.

\section{Statement of the problem}
The purpose of this study was to determine the negative effects of stress on employees
and the methods employers use to manage employees' stress.


\section{Significance of the Study}
There are three primary groups that may benefit from this study. The first group,
consisting of employees in today's business organizations, may learn to identify ways
that stress negatively affects their work performance. Identifying the negative effects
may enable them to take necessary action to cope with stress. By sharing this
knowledge, employees can act as a vehicle to help management implement appropriate
stress reduction programs. The second group that may benefit from this study is
employers who may gain insight as to how stress is actually negatively affecting
employees’ work performance. Finally, educators can use these findings as a valuable
guide to incorporate into their curriculum. By emphasizing to students the importance
of developing programs to deal with stress, the students may be able to transfer this
knowledge to the workplace, thereby improving the quality of the work environment.

\section{Scope of the study}
This study was limited to the perceptions of full-time business employees as to the
negative effects that stress has on work performance and the steps that employers are
taking to manage stress. For the purpose of this study, what constitutes full-time
employment is defined by the employer. This study was restricted to businesses
operating in the Central Texas area. The Central Texas area encompasses all
communities within Hays, Kendall, Travis, and Williamson counties. For the purpose
of this study, stress is defined as disruptive or disquieting influences that negatively
affect an individual in the workplace. Data for this study were collected during the fall
of 2002.

\section{Review of Related Literature}
Barden (2001), a freelance writer specializing in health care and a former managing
editor of Commerce and Health, stated the importance of wellness programs and gave
specific examples of corporations that are successfully implementing such
programs. The Morrison Company currently saves $8.33 for every dollar spent on
wellness by offering programs such as weight loss, exercise, and back care. Axon
Petroleum estimates that wellness programs will save $1.6 million each year in health
care costs for its 650 employees. In addition to Morrison and Axon Petroleum, Barden
cited the savings for six other companies. According to the Wellness Bureau of
America, the success of these companies offers concrete proof that wellness programs
pay off by lowering health care costs, reducing absenteeism, and increasing
productivity.
Foster (2002), a professional speaker on stress-management, surveyed midlevel
managers and found stress to be a major determinant in worker productivity.
According to the study, the primary areas affected by stress are employee morale,
absenteeism, and decision making abilities. By recognizing that a problem exists and
by addressing the issue, managers can reduce stressful activities and increase worker
performance in the business organization.
Harrold and Wayland (2002) reported that increasing stress affects morale,
productivity, organizational efficiency, absenteeism, and profitability for both
individuals and the organization. The problem for businesses today is knowing how
to determine stressful areas in their organizations and how to use constructive confrontation
methods to reduce stress and improve efficiency. According to the authors,
organizations that make a positive effort to deal with stress not only help build trust
among their employees, but also increase the productivity of their employees and the
organization as a whole.
Maurer (2002) stated that stress-induced illnesses are prevalent in the
workplace today, and stress is the problem of the sufferer and the employer. Stress
causes absenteeism and can lead to other problems such as drug addiction, alcoholism,
depression, and poor job performance. According to Maurer, the annual Barlow
Corporation Forum on Human Resource Issues and Trends reported that large numbers
of companies noticed severe levels of stress exhibited by employees. The forum's
panelists agreed that more needed to be done in the workplace to help employees
manage stress. Some of the suggestions were to expand wellness programs, offer
stress-management seminars, and teach staff how to balance work and family life.
Maurer also noted that Olympic TeamTech, a computer management company, has
dealt with employee stress by providing training programs, monitoring employee
concerns, and meeting once a month to be proactive instead of reactive. Olympic
TeamTech's turnover is less than the industry average.
Schorr (2001), a stress-management consultant, stated that stress causes
problems in the workplace which negatively affect employee health and organizational
productivity. Stress can lead to problems such as job dissatisfaction, alcoholism,
absenteeism, physical ailments, and poor job performance. If managers know how to
prevent and cope with stress, productivity can be increased. Many companies
instituted stress-management programs that led to a decline in absenteeism, a decrease
in sickness and accident costs, and/or an increase in job performance. Schorr reported
that a stress inventory, available from a stress-management program, can assist
executives and managers in assessing employee stress. The inventory can identify the
sources of stress, which may include physical elements as well as other factors. Once
these sources have been assessed, the program can provide the necessary skills for
coping with the problems, and participants can learn that there are alternative ways of
reacting to stress.

\section{Methods of the Study}
\subsection{Source of Data}
Data for this study were collected using a questionnaire developed by a group of
students at Southwest Texas State University. The questionnaire was divided intothree parts. Part one consisted of a list of 15 work performance areas that may be
negatively affected by a person's level of stress. Respondents were asked to indicate
whether stress increased, decreased, or had not changed their work performance in
each area. They were also asked to indicate from the list of 15 work performance areas
the area that was the most negatively affected by stress and the area that was the least
negatively affected by stress. In the second part of the questionnaire, a list of 17
programs was provided and the respondents were asked to indicate which programs
their companies had implemented to manage stress. Part three was designed to collect
demographic data for a respondent profile, including full-time employee classification
and age group.

\subsection{Sample Selection}
The respondents involved in this survey were employees working in companies
located in Central Texas. A non probability, convenience sampling technique was used
to collect primary data. Each member of the research team was responsible for
distributing three questionnaires to members of the sample. To ensure confidentiality,
respondents were given self-addressed, stamped envelopes in which to return their
completed questionnaires to Southwest Texas State University. Controls were used to
eliminate duplication of the responses.

\subsection{Statistical Methods}
Simple statistical techniques were used to tabulate the results of this study. The
primary data were analyzed using a percent of response. To compute the percent of
response, the number of responses to each choice was divided by the total number of
respondents who answered the question. In question one, the percents of responses for
the negative effects of stress on the 15 work performance areas were reported. The
results of the next two questions were tabulated by totaling the number of respondents
who chose an area they believed was least or most affected by stress. The fourth
question reported the percent of respondents whose employers offered the listed
programs to manage stress. Questions five and six asked the respondents to indicate if
they were considered full-time employees and to indicate their age group.

\section{Limitations Of The Study}
This study may be limited through the use of a questionnaire as a data collection
instrument. Because questionnaires must generally be brief, areas that may have been
affected by stress may not have been included in the questionnaire.
Also, all programs that may be available to employees for managing stress may
not have been included in the study. The study may also be limited by the use of a
nonprobability, convenience sampling method. The sample of business employees for
the study was chosen for convenience and may not be representative of the total
population of business employees. Care should be taken when generalizing these
findings to the entire population. Finally, the use of simple statistical techniques may
introduce an element of subjectivity into the interpretation and analysis of the data. All
attempts have been made to minimize the effects of these limitations on the study.


\end{document}